\documentclass{article}

\usepackage{german}
\usepackage[utf8]{inputenc}
\usepackage{enumerate}
\usepackage{comment}

\setlength{\textwidth}{15cm}
\setlength{\textheight}{24cm}
\addtolength{\topmargin}{-2cm}
\addtolength{\oddsidemargin}{-1.5cm}

\title{\textsf{\textbf{S A T Z U N G S Ä N D E R U N G}}\\
\small\textbf{Software Campus Alumni e.V. (SWCA e.V.)}\\
Anlage zum Protokoll der Mitgliederversammlung vom 13.01.2022}

\author{}
\date{}

\hyphenation{Software}

\begin{document}
\maketitle

Der \textbf{§2} der Satzung wird wie folgt geändert:

\begin{enumerate}[§ 2.]
\item \textsf{\textbf{Vereinszweck}}

\begin{enumerate}[1.]
	\item Der Verein verfolgt ausschließlich und unmittelbar gemeinnützige Zwecke im Sinne des Abschnitts „Steuerbegünstigte Zwecke“ der Abgabenordnung.
	Zweck des Vereins ist die Förderung von Wissenschaft und Forschung sowie der Berufsbildung und Studierendenhilfe, vor allem auf dem Gebiet der Informations- und Kommunikationstechnologie, sowie die Förderung des Gedankenaustauschs zwischen akademischer Forschung, industrieller Forschung und den Nutzern von Forschungsergebnissen.
	Der Verein kann zudem ergänzende Aufgaben übernehmen, die den Zweck zu fördern geeignet sind.

	\item Der Zweck des Vereins wird unter anderem verwirklicht durch:
		\begin{enumerate}[a.]
		\item Den Betrieb von Kommunikations- und Informationsplattformen im Internet,
			die wissenschaftliche Erkenntnisse verbreiten und den wissenschaftlichen Austausch fördern.
		\item Die Organisation und Durchführung von Veranstaltungen
			(z.B. Seminaren, Schulungen, Konferenzen, und Fachvorträgen).
		\item Den Aufbau und Ausbau von Kontakten zur interdisziplinären Vernetzung von Wissenschaft, Forschung, Lehre und Industrie.
		\item Die Förderung von Bildungs- und Forschungsprogrammen, insbesondere des Software Campus Programms.
		\item Die Öffentlichkeitsarbeit zur Bekanntmachung und Außendarstellung von
			Bildungs- und Forschungsaktivitäten, insbesondere des Software Campus Programms.
		\item Die Zusammenarbeit mit anderen Vereinen, Stiftungen und Alumni-Organisationen
			aus Bildung, Wissenschaft, Politik und Wirtschaft.
		\item Die Förderung der Gemeinschaft und des Gedankenaustauschs zwischen ehemaligen und aktuellen Teilnehmern des Software Campus.
		\item Die Beschaffung und Weiterleitung von Mitteln zur Nutzung für gemeinnützige Zwecke im Sinne des Absatz 1.
		\item Die Förderung der Berufsbildung im Hinblick auf Führungsqualifikationen durch Schulungsveranstaltungen
			und Bereitstellung von Informations- und Lehrmaterialien.
		\end{enumerate}
	\end{enumerate}
\end{enumerate}

\end{document}