\documentclass{article}

\usepackage{german}
\usepackage[utf8]{inputenc}
\usepackage{enumerate}
\usepackage{comment}

\setlength{\textwidth}{15cm}
\setlength{\textheight}{24cm}
\addtolength{\topmargin}{-2cm}
\addtolength{\oddsidemargin}{-1.5cm}

\title{\textsf{\textbf{S A T Z U N G}}\\
\small\textbf{Software Campus Alumni e.V. (SWCA e.V.)}\\
vom 09.02.2015, geändert mit Beschluss vom 25.11.2017}

\author{}
\date{}

\hyphenation{Software}

\begin{document}
\maketitle

\begin{enumerate}[§ 1.]

\item \textsf{\textbf{Name, Sitz und Geschäftsjahr}}
	\begin{enumerate}[1.]
	\item Der Verein trägt den Namen „Software Campus Alumni“ (SWCA).
	\item Nach dem Eintrag in das Vereinsregister trägt der Verein zu seinem Namen den Zusatz „eingetragener Verein“ oder „e.V.“
	\item Der Sitz des Vereins ist Berlin.
	\item Das Geschäftsjahr des Vereins ist das Kalenderjahr.
	\item Der Verein soll in das Vereinsregister eingetragen werden.
	\end{enumerate}

\item \textsf{\textbf{Vereinszweck}}
	\begin{enumerate}[1.]
	\item Der Verein verfolgt ausschließlich und unmittelbar gemeinnützige Zwecke im Sinne des Abschnitts „Steuerbegünstigte Zwecke“ der Abgabenordnung.
	Zweck des Vereins ist die Förderung von Wissenschaft und Forschung sowie der Berufsbildung und Studierendenhilfe, vor allem auf dem Gebiet der Informations- und Kommunikationstechnologie, sowie die Förderung des Gedankenaustauschs zwischen akademischer Forschung, industrieller Forschung und den Nutzern von Forschungsergebnissen.
	Der Verein kann zudem ergänzende Aufgaben übernehmen, die den Zweck zu fördern geeignet sind.

	\item Der Zweck des Vereins wird unter anderem verwirklicht durch:
		\begin{enumerate}[a.]
		\item Den Betrieb von Kommunikations- und Informationsplattformen im Internet,
			die wissenschaftliche Erkenntnisse verbreiten und den wissenschaftlichen Austausch fördern.
		\item Die Organisation und Durchführung von Veranstaltungen
			(z.B. Seminaren, Schulungen, Konferenzen, und Fachvorträgen).
		\item Den Aufbau und Ausbau von Kontakten zur interdisziplinären Vernetzung von Wissenschaft, Forschung, Lehre und Industrie.
		\item Die Förderung von Bildungs- und Forschungsprogrammen, insbesondere des Software Campus Programms.
		\item Die Öffentlichkeitsarbeit zur Bekanntmachung und Außendarstellung von
			Bildungs- und Forschungsaktivitäten, insbesondere des Software Campus Programms.
		\item Die Zusammenarbeit mit anderen Vereinen, Stiftungen und Alumni-Organisationen
			aus Bildung, Wissenschaft, Politik und Wirtschaft.
		\item Die Förderung der Gemeinschaft und des Gedankenaustauschs zwischen ehemaligen und aktuellen Teilnehmern des Software Campus.
		\item Die Beschaffung und Weiterleitung von Mitteln zur Nutzung für gemeinnützige Zwecke im Sinne des Absatz 1.
		\item Die Förderung der Berufsbildung im Hinblick auf Führungsqualifikationen durch Schulungsveranstaltungen
			und Bereitstellung von Informations- und Lehrmaterialien.
		\end{enumerate}
	\end{enumerate}

\item \textsf{\textbf{Tätigkeitsgrundsätze}}
	\begin{enumerate}[1.]
	\item Der Verein ist politisch, weltanschaulich und konfessionell neutral.
	\item Der Verein dient ausschließlich den unter § 2 aufgeführten Zwecken. Er ist
selbstlos tätig und verfolgt nicht in erster Linie eigenwirtschaftliche Zwecke. 
	\item Die Mittel des Vereins dürfen nur für satzungsmäßige Zwecke verwendet werden.
Niemand darf durch Ausgaben, die dem Zweck des Vereins fremd sind, oder durch
unverhältnismäßige Vergütungen begünstigt werden.
	\item Sämtliche Mitglieder der Organe des Vereins üben ihre Tätigkeit ehrenamtlich aus.
Die Mitglieder des Vereins erhalten in Ihrer Eigenschaft als Mitglieder keine Zuwendungen aus den Vereinsmitteln
und haben keinen Anteil am Vereinsvermögen.
Die im Interesse des Vereins entstandenen Reisekosten und Tagegelder werden
in der vom Vorstand festgesetzten Höhe ersetzt.
Die Mitgliederversammlung kann die vom Vorstand festzusetzende Höhe der Reisekosten und Tagegeldern durch Beschluss begrenzen.
	\item Bei Bedarf können Vereinsämter im Rahmen der haushaltsrechtlichen Möglichkeiten abweichend von Absatz 4 auch entgeltlich auf der Grundlage eines Dienstvertrages oder gegen Zahlung einer Aufwandsentschädigung nach § 3 Nr. 26 EStG ausgeübt werden.
Die Entscheidung über eine entgeltliche Vereinstätigkeit trifft die Mitgliederversammlung auf Beschlussvorschlag des Vorstands oder eines Mitglieds.
Der Beschlussvorschlag umfasst die Vertragsinhalte oder die Vertragsbeendigung.
	\item Der Verein kann im gemeinnützigkeitsrechtlich zulässigen Rahmen Rücklagen bilden.
Dies gilt insbesondere für geplante Veranstaltungen, Betriebsmittelrücklagen und Personalmittelrücklagen die erforderlich sind, um die steuerbegünstigten satzungsgemäßen Zwecke des Vereins nachhaltig zu erfüllen.
	\end{enumerate}

\item \textsf{\textbf{Mitglieder}}
	\begin{enumerate}[1.]
	\item Mitglieder des Vereins sind ordentliche Mitglieder, Fördermitglieder und
Ehrenmitglieder.
	\item Ordentliche Mitglieder des Vereins können natürliche Personen werden, die am
Software Campus teilnehmen oder teilgenommen haben.
Ordentliche Mitglieder
dürfen an der Mitgliederversammlung teilnehmen und ihr Stimmrecht ausüben.
	\item Fördermitglieder können natürliche Personen, juristische Personen oder
Personengesellschaften sein, die den Verein unterstützen wollen. Fördermitglieder
können an der Mitgliederversammlung ohne Stimmrecht teilnehmen.
	\item Über die Ernennung zum Ehrenmitglied entscheidet die Mitgliederversammlung auf Vorschlag des Vorstandes.
Ehrenmitglieder können an der Mitgliederversammlung ohne Stimmrecht teilnehmen.
	\item Die Mitglieder haben die Pflicht gemäß § 7, die von der Mitgliederversammlung
beschlossenen Beiträge zu zahlen.
	\item Die Mitglieder haben die Pflicht sich an den von der Mitgliederversammlung beschlossenen Verhaltenskodex zu halten.
	\item Die Mitglieder haben die Pflicht dem Verein stets aktuelle Kontaktdaten, mindestens eine Postanschrift und eine E-Mail Adresse, mitzuteilen. 
	\end{enumerate}

\item \textsf{\textbf{Aufnahme und Beginn der Mitgliedschaft}}
	\begin{enumerate}[1.]
	\item Über die Aufnahme neuer Mitglieder entscheidet der Vorstand.
	Die Mitgliederversammlung darf durch Beschluss Regeln festlegen an die der Vorstand bei der Entscheidung über die Aufnahme neuer Mitglieder gebunden ist.
	Ein Anspruch auf Mitgliedschaft besteht nicht.
	Der Vorstand ist berechtigt die Entscheidung über die Aufnahme eines neuen Mitglieds an die Mitgliederversammlung zu delegieren.
	In diesem Fall entscheidet die Mitgliederversammlung mit einfacher Mehrheit über die Aufnahme.
	\item Lehnt der Vorstand die Aufnahme ab, steht dem/der Betroffenen die Berufung an die Mitgliederversammlung zu.
	Diese kann die Entscheidung des Vorstands mit einer 3/4-Mehrheit der abgegebenen Stimmen überstimmen.
	\item Ein neues Mitglied gilt erst dann als aufgenommen, wenn die Aufnahme nach Absatz 1 oder 2 erfolgt ist und der fällige Mitgliedsbeitrag gezahlt wurde.
	\end{enumerate}

\item \textsf{\textbf{Beendigung der Mitgliedschaft}}
	\begin{enumerate}[1.]
	\item Die Mitgliedschaft wird beendet durch:
		\begin{enumerate}[a.]
		\item Austritt nach schriftlicher Kündigung beim Vorstand. Bezahlte
Beiträge werden nicht erstattet.
		\item Tod des Mitglieds.
		\item Ausschluss des Mitglieds. Ein Mitglied kann ausgeschlossen werden, wenn es seinen guten Ruf verliert, das Ansehen des Vereins in erheblichem Ausmaß schädigt, gegen den von der Mitgliederversammlung beschlossenen Verhaltenskodex verstößt, oder dem Verein materiellen Schaden zufügt.
		\item Einen Ausschluss kann entweder die Mitgliederversammlung mit 3/4 der abgegebenen Stimmen oder  der Vorstand beschließen.
Dem Mitglied ist vorher Gelegenheit zur Äußerung zu geben.
Der Ausschließungsbeschluss ist dem Mitglied unter Bekanntgabe der Gründe durch einen eingeschriebenen Brief mit Rückschein bekannt zu geben.
Dem Mitglied steht gegen einen Vorstandsbeschluss das Recht der Berufung zur nächsten Mitgliederversammlung zu.
Die Vorstandsentscheidung kann von der Mitgliederversammlung mit 3/4-Mehrheit der abgegebenen Stimmen überstimmt werden.
Der Vorstand kann den Ausschluss desselben Mitglieds nur dann erneut beschließen, wenn weitere Ausschlussgründe eingetreten sind, die deutlich von den vorherigen Ausschlussgründen abweichen oder über diese hinaus gehen.
		\item Nichtzahlung des Mitgliedsbeitrages binnen 3 Monaten nach Zahlungserinnerung, wenn der Vorstand den Ausschluss mit einfacher Mehrheit beschließt. Wenn das Mitglied unter den dem Verein bekannten Kontaktdaten nicht erreichbar ist, kann der Vorstand den Ausschluss auch ohne Zahlungserinnerung beschließen, wenn der Beitrag mehr als 3 Monate überfällig ist.
		\end{enumerate}
	\item Die offenen Forderungen gegen Mitglieder erlöschen durch Beendigung der Mitgliedschaft nicht.
	Auch nach Beendigung sind offene Forderungen von dem entsprechenden Mitglied zu begleichen.
	\end{enumerate}

\item \textsf{\textbf{7 Mitgliedsbeitrag}}
	\begin{enumerate}[1.]
	\item Die Mitgliedsbeiträge sind Jahresbeiträge und jeweils am 1. Januar eines Jahres
im Voraus fällig.
	\item Über die Höhe der unterschiedlichen Jahresbeiträge entscheidet die
Mitgliederversammlung.
	\item Ehrenmitglieder und Fördermitglieder sind von der Beitragszahlung befreit.
	\end{enumerate}
	
\item \textsf{\textbf{Organe des Vereins}}

Die Organe des Vereins sind:
	\begin{enumerate}[a.]
	\item Der Vorstand.
	\item Die Vorstandschaft.
	\item Der Beirat.
	\item Die Mitgliederversammlung.
	\end{enumerate}

\item \textsf{\textbf{Vorstand}}
	\begin{enumerate}[1.]
	\item Der Vorstand im Sinne des § 26 BGB besteht aus der/dem 1. und 2. Vorsitzenden.
Beide Vorstandsmitglieder sind einzeln vertretungsberechtigt.
	\item Die Vertretungsbefugnis der Vorstandsmitglieder ist in der Weise beschränkt, dass sie zu Rechtsgeschäften im Wert von mehr als 5.000,00 Euro der Zustimmung der Vorstandschaft bedürfen und zu Grundstücksgeschäften die Zustimmung der Mitgliederversammlung erforderlich ist.
	\item Der Vorstand wird von der Mitgliederversammlung auf die Dauer von 2 Jahren gewählt. Gewählt ist, wer die einfache Mehrheit der Stimmen der anwesenden Mitglieder auf sich vereint. Eine Wiederwahl ist möglich. Er bleibt bis zur satzungsmäßigen Bestellung eines neuen Vorstands im Amt. Scheidet ein Mitglied des Vorstands während einer Amtsperiode aus, wählt die Vorstandschaft ein Ersatzmitglied bis zur nächsten Mitgliederversammlung.
	\end{enumerate}

\item \textsf{\textbf{Vorstandschaft}}
	\begin{enumerate}[1.]
	\item Die Vorstandschaft besteht aus:
		\begin{enumerate}[a.]
		\item Dem Vorstand gemäß § 9.
		\item Dem/der Ressortleiter/-in Finanzen/Schatzmeister/-in
		\item Den restlichen Ressortleitern
		\item Vertreter des Beirats gemäß § 11
		\end{enumerate}
	\item Die Vorstandschaft fasst ihre Beschlüsse in Sitzungen, die vom Vorstand mit einer
Frist von mindestens 14 Tagen einberufen werden und ist beschlussfähig, wenn
mindestens die Hälfte ihrer Mitglieder teilnehmen. Sie fasst ihre Beschlüsse mit
einfacher Stimmenmehrheit. Bei Stimmengleichheit gilt der Antrag als abgelehnt.
	\item Die Aufgabengebiete der Ressorts und deren Leiter werden nach den jeweiligen
Erfordernissen nach § 12 dieser Satzung von der Mitgliederversammlung in
Vereinsordnungen festgelegt.
	\item Die Vorstandschaft wird analog dem Vorstand von der Mitgliederversammlung auf
die Dauer von 2 Jahren gewählt. Eine Wiederwahl ist möglich. Sie bleibt bis zur
satzungsmäßigen Bestellung eines neuen Gremiums im Amt.
	\item Eine Person darf mehrere der oben aufgeführten Ämter auf sich vereinen. Davon
ausgeschlossen ist die Vereinigung von 1. oder 2. Vorsitzenden mit dem/der
Ressortleiter/-in Finanzen/Schatzmeister/-in in einer Person.
	\item Der/die Schatzmeister/-in/Ressortleiter/-in Finanzen hat die Beiträge der Mitglieder
einzuziehen und das Vermögen des Vereins zu verwalten. Er/Sie erstattet in der
ordentlichen Mitgliederversammlung seinen Rechenschaftsbericht.
	\end{enumerate}

\item \textsf{\textbf{Beirat}}
	\begin{enumerate}[1.]
	\item Der Beirat hat die Aufgabe dem Verein bei der Verfolgung seiner
satzungsmäßigen Zwecke beratend zur Seite zu stehen.
	\item Der Beirat besteht aus den Mitgliedern des Lenkungsausschusses des Software
Campus.
	\item Der Beirat darf 2 seiner Mitglieder bestimmen, die ihn bei der Vorstandschaft
vertreten.
	\item Sofern der Beirat Vertreter bestimmt hat, sind diese vom geschäftsführenden
Vorstand über aktuelle Vorhaben und Ideen zu unterrichten.
	\item Der Beirat kann Vorschläge über die Vorstandschaft einbringen, welche dann den
Mitgliedern zur Abstimmung vorgelegt werden.
	\end{enumerate}

\item \textsf{\textbf{Mitgliederversammlung}}
	\begin{enumerate}[1.]
	\item Einmal jährlich ist vom Vorstand eine ordentliche Mitgliederversammlung
einzuberufen.
	Die Ladung hierzu hat mit einer Frist von in der Regel mindestens 30 Tagen, in begründeten Ausnahmefällen von mindestens 14 Tagen persönlich durch einfachen Brief oder via E-Mail mit Bekanntgabe der Tagesordnung zu erfolgen. Letztere ist von der Vorstandschaft festzulegen.
	\item Jedes ordentliche Vereinsmitglied hat eine Stimme. Zur Ausübung des
	Stimmrechts kann ein anderes Mitglied schriftlich bevollmächtigt werden.
	Niemand darf mehr als insgesamt 2 Stimmen abgeben.
	\item Anträge der Mitglieder zur Tagesordnung sind bis spätestens 5 Tage vor der Versammlung beim 1. Vorsitzenden schriftlich, auch per E-Mail, einzureichen.
	\item Außerordentliche Mitgliederversammlungen haben stattzufinden, wenn solches die Vorstandschaft beschließt oder mindestens ein Zehntel der Mitglieder unter Nennung der Gründe es verlangen. Soweit nicht anderweitig spezifiziert, gelten die Regelungen der ordentlichen Mitgliederversammlung analog.
	\item Die Mitgliederversammlung ist ohne Rücksicht auf die Zahl der Teilnehmenden Mitglieder beschlussfähig, ausgenommen über einen Beschluss über die Auflösung des Vereins oder die Änderung des Zwecks.
	Eine Vereinsauflösung oder Änderung des Zwecks kann nur erfolgen, wenn diese in der Einladung angekündigt ist und mindestens 3/4 der stimmberechtigten Mitglieder anwesend sind.
	Wird diese Zahl nicht erreicht, kann zum Zweck der Auflösung oder der Änderung des gemeinnützigen Zwecks in frühestens 21 Tagen eine weitere Mitgliederversammlung stattfinden, die sodann ohne Rücksicht auf die Zahl der anwesenden Mitglieder beschlussfähig ist.
	In der Einladung zu dieser Versammlung ist darauf besonders hinzuweisen.
	\item Den Vorsitz der Mitgliederversammlung führt der/die 1., bei dessen/deren Verhinderung der/die 2., Vorsitzende. Sind beide Personen nicht anwesend, bestimmt die Mitgliederversammlung eine/n Versammlungsleiter/-in aus ihrer Mitte.
	Alle Abstimmungen und Wahlen können per Akklamation erfolgen, es sei denn, dass mindestens ein Zehntel der anwesenden, stimmberechtigten Mitglieder geheime Abstimmung verlangt.
	Zur Durchführung von Wahlen ist ein Wahlausschuss zu bilden, dem mindestens 2 Vereinsmitglieder, die nicht selbst kandidieren, angehören müssen. Dies können auch Fördermitglieder sein.
	\item Über die Mitgliederversammlung ist ein Protokoll zu führen. Der/Die
Protokollführer/-in muss ordentliches Mitglied des Vereins sein und wird vom/von der Versammlungsleiter/-in ernannt.
	Das Protokoll bestätigt die Ordnungsmäßigkeit der Versammlung und enthält alle Beschlüsse.
	Es ist vom/von	der Versammlungsleiter/-in und vom/von der	Protokollführer/-in zu unterzeichnen und wird den Mitgliedern zugänglich gemacht.
	\item Der Mitgliederversammlung obliegt mit einfacher Mehrheit der abgegebenen Stimmen mindestens über folgendes zu entscheiden:
		\begin{enumerate}[a.]
		\item Genehmigung der Tätigkeitsberichte und des Rechnungsabschlusses.
		\item Entlastung der Vorstandschaft.
		\item Entlastung der Kassenprüfer/-in.
		\item Festsetzung der Mitgliedsbeiträge.
		\item Wahl des Vorstandes und der Vorstandschaft.
		\end{enumerate}
	\item Eine Mehrheit von 3/4 der abgegebenen Stimmen ist erforderlich für:
		\begin{enumerate}[a.]
		\item Aufnahme weiterer Punkte in die Tagesordnung, ausgenommen
Auflösung des Vereins und Änderung des gemeinnützigen Zwecks.
		\item Erlass von Vereinsordnungen und Arbeitsanweisungen für die
Vereinsführung.
		\item Änderungen in der Satzung.
		\item Berufung über die Aufnahme und Ausschluss von Mitgliedern.
		\item Auflösung des Vereins und Verwendung des restlichen
Vereinsvermögens.
		\end{enumerate}
	\end{enumerate}

\item \textsf{\textbf{Auflösung des Vereins und Zweckänderung}}
	\begin{enumerate}[1.]
	\item Die Auflösung des Vereins oder die Änderung seines gemeinnützigen Zwecks
kann nur in einer dazu einberufenen Mitgliederversammlung erfolgen. Sofern
diese nicht besondere Liquidatoren bestellt, sind der 1. und 2. Vorsitzende
gemeinsam vertretungsberechtigte Liquidatoren. Änderungen des Vereinszwecks
bedürfen der Genehmigung des Finanzamtes.
	\item Das Vermögen fällt bei seiner Auflösung einer im Rahmen des
Auflösungsbeschlusses zu bestimmenden gemeinnützigen Organisation zu,
deren Zweck den Zwecken des Software Campus Alumni Vereins nahe steht.
	\end{enumerate}

\end{enumerate}

\end{document}