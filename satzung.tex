\documentclass{article}

\usepackage{german}
\usepackage[utf8]{inputenc}
\usepackage{enumerate}
\usepackage{comment}

\setlength{\textwidth}{15cm}
\setlength{\textheight}{24cm}
\addtolength{\topmargin}{-2cm}
\addtolength{\oddsidemargin}{-1.5cm}

\title{\textsf{\textbf{S A T Z U N G}}\\
\small\textbf{Software Campus Alumni e.V. (SWCA e.V.)}\\
Neufassung 2021 (Beschlussvorlage)}

\author{}
\date{}

\hyphenation{Software}

\begin{document}
\maketitle

\begin{enumerate}[§ 1.]

\item \textsf{\textbf{Name, Sitz und Geschäftsjahr}}
	\begin{enumerate}[1.]
	\item Der Verein trägt den Namen „Software Campus Alumni“ (SWCA).
	\item Nach dem Eintrag in das Vereinsregister trägt der Verein zu seinem Namen den Zusatz „eingetragener Verein“ oder „e.V.“
	\item Der Sitz des Vereins ist Berlin.
	\item Das Geschäftsjahr des Vereins ist das Kalenderjahr.
	\item Der Verein soll in das Vereinsregister eingetragen werden.
	\end{enumerate}

\item \textsf{\textbf{Vereinszweck}}
	\begin{enumerate}[1.]
	\item Der Verein verfolgt ausschließlich und unmittelbar gemeinnützige Zwecke im Sinne des Abschnitts „Steuerbegünstigte Zwecke“ der Abgabenordnung.
	Zweck des Vereins ist die Förderung von Wissenschaft und Forschung sowie der Berufsbildung und Studierendenhilfe, vor allem auf dem Gebiet der Informations- und Kommunikationstechnologie, sowie die Förderung des Gedankenaustauschs zwischen akademischer Forschung, industrieller Forschung und den Nutzern von Forschungsergebnissen.
	Der Verein kann zudem ergänzende Aufgaben übernehmen, die den Zweck zu fördern geeignet sind.

	\item Der Zweck des Vereins wird unter anderem verwirklicht durch:
		\begin{enumerate}[a.]
		\item Den Betrieb von Kommunikations- und Informationsplattformen im Internet,
			die wissenschaftliche Erkenntnisse verbreiten und den wissenschaftlichen Austausch fördern.
		\item Die Organisation und Durchführung von Veranstaltungen
			(z.B. Seminaren, Schulungen, Konferenzen, und Fachvorträgen).
		\item Den Aufbau und Ausbau von Kontakten zur interdisziplinären Vernetzung von Wissenschaft, Forschung, Lehre und Industrie.
		\item Die Förderung von Bildungs- und Forschungsprogrammen, insbesondere des Software Campus Programms.
		\item Die Öffentlichkeitsarbeit zur Bekanntmachung und Außendarstellung von
			Bildungs- und Forschungsaktivitäten, insbesondere des Software Campus Programms.
		\item Die Zusammenarbeit mit anderen Vereinen, Stiftungen und Alumni-Organisationen
			aus Bildung, Wissenschaft, Politik und Wirtschaft.
		\item Die Förderung der Gemeinschaft und des Gedankenaustauschs zwischen ehemaligen und aktuellen Teilnehmern des Software Campus.
		\item Die Beschaffung und Weiterleitung von Mitteln zur Nutzung für gemeinnützige Zwecke im Sinne des Absatz 1.
		\item Die Förderung der Berufsbildung im Hinblick auf Führungsqualifikationen durch Schulungsveranstaltungen
			und Bereitstellung von Informations- und Lehrmaterialien.
		\end{enumerate}
	\end{enumerate}

\item \textsf{\textbf{Tätigkeitsgrundsätze}}
	\begin{enumerate}[1.]
	\item Der Verein ist politisch, weltanschaulich und konfessionell neutral.
	\item Der Verein dient ausschließlich den unter § 2 aufgeführten Zwecken. Er ist
selbstlos tätig und verfolgt nicht in erster Linie eigenwirtschaftliche Zwecke. 
	\item Die Mittel des Vereins dürfen nur für satzungsmäßige Zwecke verwendet werden.
Niemand darf durch Ausgaben, die dem Zweck des Vereins fremd sind, oder durch
unverhältnismäßige Vergütungen begünstigt werden.
	\item Sämtliche Mitglieder der Organe des Vereins üben ihre Tätigkeit ehrenamtlich aus.
Die Mitglieder des Vereins erhalten in Ihrer Eigenschaft als Mitglieder keine Zuwendungen aus den Vereinsmitteln
und haben keinen Anteil am Vereinsvermögen.
Die im Interesse des Vereins entstandenen Reisekosten und Tagegelder werden
in der vom Vorstand festgesetzten Höhe ersetzt.
Die Mitgliederversammlung kann die vom Vorstand festzusetzende Höhe der Reisekosten und Tagegeldern durch Beschluss begrenzen.
	\item Bei Bedarf können Vereinsämter im Rahmen der haushaltsrechtlichen Möglichkeiten abweichend von Absatz 4 auch entgeltlich auf der Grundlage eines Dienstvertrages oder gegen Zahlung einer Aufwandsentschädigung nach § 3 Nr. 26 EStG ausgeübt werden.
Die Entscheidung über eine entgeltliche Vereinstätigkeit trifft die Mitgliederversammlung auf Beschlussvorschlag des Vorstands oder eines Mitglieds.
Der Beschlussvorschlag umfasst die Vertragsinhalte oder die Vertragsbeendigung.
	\item Der Verein kann im gemeinnützigkeitsrechtlich zulässigen Rahmen Rücklagen bilden.
Dies gilt insbesondere für geplante Veranstaltungen, Betriebsmittelrücklagen und Personalmittelrücklagen die erforderlich sind, um die steuerbegünstigten satzungsgemäßen Zwecke des Vereins nachhaltig zu erfüllen.
	\end{enumerate}

\item \textsf{\textbf{Mitglieder}}
	\begin{enumerate}[1.]
	\item Mitglieder des Vereins sind ordentliche Mitglieder, Fördermitglieder und
Ehrenmitglieder.
	\item Ordentliche Mitglieder des Vereins können natürliche Personen werden, die am
Software Campus teilnehmen oder teilgenommen haben.
Ordentliche Mitglieder
dürfen an der Mitgliederversammlung teilnehmen und ihr Stimmrecht ausüben.
	\item Fördermitglieder können natürliche Personen, juristische Personen oder
Personengesellschaften sein, die den Verein unterstützen wollen. Fördermitglieder
können an der Mitgliederversammlung ohne Stimmrecht teilnehmen.
	\item Über die Ernennung zum Ehrenmitglied entscheidet die Mitgliederversammlung auf Vorschlag des Vorstandes.
Ehrenmitglieder können an der Mitgliederversammlung ohne Stimmrecht teilnehmen.
	\item Die Mitglieder haben die Pflicht gemäß § 7, die von der Mitgliederversammlung
beschlossenen Beiträge zu zahlen.
	\item Die Mitglieder haben die Pflicht sich an den von der Mitgliederversammlung beschlossenen Verhaltenskodex zu halten.
	\item Die Mitglieder haben die Pflicht dem Verein stets aktuelle Kontaktdaten, mindestens eine Postanschrift und eine E-Mail Adresse, mitzuteilen. 
	\end{enumerate}

\item \textsf{\textbf{Aufnahme und Beginn der Mitgliedschaft}}
	\begin{enumerate}[1.]
	\item Über die Aufnahme neuer Mitglieder entscheidet der Vorstand.
	Die Mitgliederversammlung darf durch Beschluss Regeln festlegen an die der Vorstand bei der Entscheidung über die Aufnahme neuer Mitglieder gebunden ist.
	Ein Anspruch auf Mitgliedschaft besteht nicht.
	Der Vorstand ist berechtigt die Entscheidung über die Aufnahme eines neuen Mitglieds an die Mitgliederversammlung zu delegieren.
	In diesem Fall entscheidet die Mitgliederversammlung mit einfacher Mehrheit über die Aufnahme.
	\item Lehnt der Vorstand die Aufnahme ab, steht dem/der Betroffenen die Berufung an die Mitgliederversammlung zu.
	Diese kann die Entscheidung des Vorstands mit einer 3/4-Mehrheit der abgegebenen Stimmen überstimmen.
	\item Ein neues Mitglied gilt erst dann als aufgenommen, wenn die Aufnahme nach Absatz 1 oder 2 erfolgt ist und der fällige Mitgliedsbeitrag gezahlt wurde.
	\end{enumerate}

\item \textsf{\textbf{Beendigung der Mitgliedschaft}}
	\begin{enumerate}[1.]
	\item Die Mitgliedschaft wird beendet durch:
		\begin{enumerate}[a.]
		\item Austritt durch schriftliche Kündigung des Mitglieds gerichtet an den Vorstand.
		Die Mitgliedschaft endet, sofern das Mitglied keinen späteren Endzeitpunkt wünscht,
		zum Ende des Monats, der auf den Eingang der Kündgung folgt.
		Bezahlte Beiträge werden nicht erstattet.
		\item Tod des Mitglieds.
		\item Ausschluss des Mitglieds. Ein Mitglied kann ausgeschlossen werden, wenn es seinen guten Ruf verliert, das Ansehen des Vereins in erheblichem Ausmaß schädigt, gegen den von der Mitgliederversammlung beschlossenen Verhaltenskodex verstößt, oder dem Verein materiellen Schaden zufügt.
		\item Einen Ausschluss kann entweder die Mitgliederversammlung mit 3/4 der abgegebenen Stimmen oder  der Vorstand beschließen.
Dem Mitglied ist vorher Gelegenheit zur Äußerung zu geben.
Der Ausschließungsbeschluss ist dem Mitglied unter Bekanntgabe der Gründe durch einen eingeschriebenen Brief mit Rückschein bekannt zu geben.
Dem Mitglied steht gegen einen Vorstandsbeschluss das Recht der Berufung zur nächsten Mitgliederversammlung zu.
Die Vorstandsentscheidung kann von der Mitgliederversammlung mit 3/4-Mehrheit der abgegebenen Stimmen überstimmt werden.
Der Vorstand kann den Ausschluss desselben Mitglieds nur dann erneut beschließen, wenn weitere Ausschlussgründe eingetreten sind, die deutlich von den vorherigen Ausschlussgründen abweichen oder über diese hinaus gehen.
		\item Nichtzahlung des Mitgliedsbeitrages binnen 3 Monaten nach Zahlungserinnerung, wenn der Vorstand den Ausschluss mit einfacher Mehrheit beschließt. Wenn das Mitglied unter den dem Verein bekannten Kontaktdaten nicht erreichbar ist, kann der Vorstand den Ausschluss auch ohne Zahlungserinnerung beschließen, wenn der Beitrag mehr als 3 Monate überfällig ist.
		\end{enumerate}
	\item Die offenen Forderungen gegen Mitglieder erlöschen durch Beendigung der Mitgliedschaft nicht.
	Auch nach Beendigung sind offene Forderungen von dem entsprechenden Mitglied zu begleichen.
	\end{enumerate}
	
\item \textsf{\textbf{Mitgliedsbeitrag}}
	\begin{enumerate}[1.]
	\item Die Mitgliedsbeiträge sind Jahresbeiträge und jeweils am 1. Januar eines Jahres
im Voraus fällig.
	\item Über die Höhe der unterschiedlichen Jahresbeiträge entscheidet die
Mitgliederversammlung.
	\item Ehrenmitglieder sind von der Beitragszahlung befreit.
	\item Für Fördermitglieder legt der Vorstand einen individuellen Beitrag fest.
	\item Der Vorstand kann einzelne Mitglieder, auf deren Antrag hin, in Härtefällen vorübergehend von der Beitragszahlung befreien und Beitragsforderungen erlassen.
	Als Härtefall gelten insbesondere Arbeitslosigkeit, Krankheit, Zahlungsunfähigkeit und andere Umstände, die ein Mitglied in finanzielle Not versetzen.
	Ein Anspruch auf Befreiung von Beitragszahlungen oder die Erlassung von Beitragsforderungen besteht nicht.
	Der Antrag kann formlos, auch mündlich, erfolgen.
	Der Vorstand protokolliert den Antrag und die Entscheidung.
	\end{enumerate}
	
\item \textsf{\textbf{Organe des Vereins}}

Die Organe des Vereins sind:
	\begin{enumerate}[a.]
	\item Der Vorstand.
	\item Die Mitgliederversammlung.
	\end{enumerate}

\item \textsf{\textbf{Vorstand}}
	\begin{enumerate}[1.]	
	\item Der Vorstand im Sinne des § 26 BGB besteht aus der/dem 1. und 2. Vorsitzenden und dem/der Kassenwart/-in.
	Die Vorstandsmitglieder sind einzeln vertretungsberechtigt.
	\item Die Vertretungsbefugnis der Vorstandsmitglieder ist in der Weise beschränkt, dass sie zu Rechtsgeschäften, welche einen von der Mitgliederversammlung festgelegten Grenzwert übersteigen, der Zustimmung der Mitgliederversammlung bedürfen und zu Grundstücksgeschäften die Zustimmung der Mitgliederversammlung erforderlich ist.
	\item Der Vorstand wird von der Mitgliederversammlung auf die Dauer von 2 Jahren gewählt.
	Gewählt ist, wer die einfache Mehrheit der Stimmen der anwesenden Mitglieder auf sich vereint.
	Eine Wiederwahl ist möglich. Sie/Er bleibt bis zur satzungsmäßigen Bestellung eines neuen Vorstands im Amt.
	Scheidet ein Mitglied des Vorstands während einer Amtsperiode aus,
	ist innerhalb von drei Monaten eine Mitgliederversammlung zur Wahl eines Ersatzmitglieds einzuberufen.
	\item Der Vorstand trifft seine Beschlüsse mit einfacher Mehrheit der Vorstandsmitglieder. Beschlüsse können in Vorstandssitzungen oder im Umlaufverfahren getroffen werden.
	\item Der Vorstand kann der Mitgliederversammlung unter Benennung der Aufgabengebiete die Schaffung von Ressorts mit eigenem Budget und die Wahl von zugehörigen Ressortleiter/-innen vorschlagen. Die Ressortleiter/-innen berichten an den Vorstand. Der Vorstand ist gegenüber den Ressortleiter/-innen weisungsbefugt. Ressortleiter/-innen müssen Mitglieder oder Mitarbeiter/-innen des Vereins sein. Sie sind nicht Teil des Vorstands. Der Vorstand informiert die Ressortleiter über seine Arbeit und gefasste Beschlüsse.  
	\item Der/die Kassenwart/-in hat die Beiträge der Mitglieder einzuziehen und das Vermögen des Vereins zu verwalten. Er/Sie erstattet in der ordentlichen Mitgliederversammlung seinen Rechenschaftsbericht.
	\item Keine Person darf mehrere der oben aufgeführten Ämter auf sich vereinen. Davon ausgeschlossen ist die Vereinigung eines Vorstandsamt mit (einer) Ressortleitung(en).
	\end{enumerate}

\item \textsf{\textbf{Mitgliederversammlung}}
	\begin{enumerate}[1.]
	\item Einmal jährlich ist vom Vorstand eine ordentliche Mitgliederversammlung
einzuberufen.
	\item Darüber hinaus ist eine eine außerordentliche Mitgliederversammlung vom Vorstand zeitnah einzuberufen, wenn das Interesse des Vereins es erfordert, insbesondere wenn mindestens 10\% der stimmberechtigten Mitglieder es verlangen. Der Antrag mit den Namen der Unterstützer ist mit Angabe des Anlasses in Textform zu dokumentieren und dem Vorstand zuzustellen.
	\item Die Ladung zu einer Mitgliederversammlung hat mit einer Frist von in der Regel mindestens 30 Tagen, in begründeten Ausnahmefällen von mindestens 14 Tagen durch einfachen Brief oder via E-Mail mit Bekanntgabe der Tagesordnung zu erfolgen. Letztere ist vom Vorstand festzulegen. Die Versammlung kann auch online und/oder dezentral stattfinden.
	\item Jedes ordentliche Vereinsmitglied, dass alle fälligen Vereinsbeiträge bezahlt hat, hat eine Stimme. Zur Ausübung des Stimmrechts kann ein anderes Mitglied bevollmächtigt werden. Bevollmächtigungen sind dem Vorstand schriftlich, per E-Mail oder über ein vom Verein bereitgestelltes Mitgliederportal im Internet bis spätestens 5 Tage vor der Versammlung anzuzeigen. Kein Mitglied darf durch Bevollmächtigungen mehr als 10\% der stimmberechtigten Mitglieder repräsentieren. Bevollmächtigungen gelten stets nur für eine Mitgliederversammlung, sowie für eine weitere Mitgliederversammlung nach Absatz 6. 
	\item Anträge der Mitglieder zur Tagesordnung sind bis spätestens 5 Tage vor der Versammlung beim 1. Vorsitzenden schriftlich, auch per E-Mail, oder über ein vom Verein bereitgestelltes Mitgliederportal im Internet, einzureichen.
	\item Die Mitgliederversammlung ist ohne Rücksicht auf die Zahl der Teilnehmenden Mitglieder beschlussfähig, ausgenommen über einen Beschluss über die Auflösung des Vereins oder die Änderung des Zwecks. Eine Vereinsauflösung oder Änderung des Zwecks kann nur erfolgen, wenn diese in der Einladung angekündigt ist und mindestens 3/4 der stimmberechtigten Mitglieder anwesend sind oder ein anwesendes Mitglied zur Stimmabgabe bevollmächtigt haben. Wird diese Zahl nicht erreicht, kann zum Zweck der Auflösung oder der Änderung des Zwecks in frühestens 21 Tagen eine weitere Mitgliederversammlung stattfinden, die sodann ohne Rücksicht auf die Zahl der anwesenden Mitglieder beschlussfähig ist. In der Einladung zu dieser Versammlung ist darauf besonders hinzuweisen. Die Einladung muss in diesem Fall mindestens 14 Tage vor der Versammlung erfolgen.
	\item Die Leitung der Mitgliederversammlung hat der/die 1., bei dessen/deren Verhinderung der/die 2., Vorsitzende inne. Sind beide Personen nicht anwesend, bestimmt die Mitgliederversammlung eine/n Versammlungsleiter/-in aus ihrer Mitte.
	\item Alle Abstimmungen und Wahlen können per Akklamation erfolgen, es sei denn, dass mindestens ein Zehntel der anwesenden, stimmberechtigten Mitglieder geheime Abstimmung verlangt.
	\item Zur Durchführung von Wahlen ist ein Wahlausschuss zu bilden, dem mindestens 2 Vereinsmitglieder, die nicht selbst kandidieren, angehören müssen. Dies können auch Mitglieder ohne Stimmrecht sein.
	\item Über die Mitgliederversammlung ist ein Protokoll zu führen. Der/Die Protokollführer/-in muss Mitglied oder Mitarbeiter/-in des Vereins sein und wird vom/von der Versammlungsleiter/-in ernannt. Das Protokoll bestätigt die Ordnungsmäßigkeit der Versammlung und enthält alle Beschlüsse. Es ist vom/von der Versammlungsleiter/-in und vom/von der Protokollführer/-in zu unterzeichnen und wird den Mitgliedern zugänglich gemacht.
	\item Der Mitgliederversammlung obliegt mit einfacher Mehrheit der abgegebenen Stimmen mindestens über folgendes zu entscheiden:
		\begin{enumerate}[a.]
		\item Genehmigung der Tätigkeitsberichte und des Rechnungsabschlusses.
		\item Entlastung des Vorstands.
		\item Entlastung der Kassenprüfer/-in.
		\item Festsetzung der Mitgliedsbeiträge.
		\item Wahl des Vorstandes.
		\end{enumerate}
	\item Eine Mehrheit von 3/4 der abgegebenen Stimmen ist erforderlich für:
		\begin{enumerate}[a.]
		\item Aufnahme weiterer Punkte in die Tagesordnung. Nicht nachträglich in die Tagesordnung aufgenommen werden können Punkte, die die Auflösung des Vereins oder die Änderung des Vereinszwecks beinhalten.
		\item Erlass von Vereinsordnungen und Arbeitsanweisungen für den Vorstand und die Ressortleitung.
		\item Änderungen in der Satzung.
		\item Berufung über die Aufnahme und Ausschluss von Mitgliedern.
		\item Auflösung des Vereins und Verwendung des restlichen
Vereinsvermögens.
		\item Die Abwahl des Vorstands.
		\end{enumerate}
	\end{enumerate}

\item \textsf{\textbf{Auflösung des Vereins und Zweckänderung}}
	\begin{enumerate}[1.]
	\item Die Auflösung des Vereins oder die Änderung seines Zwecks
kann nur in einer dazu einberufenen Mitgliederversammlung erfolgen. Sofern
diese nicht besondere Liquidatoren bestellt, sind der 1. und 2. Vorsitzende
gemeinsam vertretungsberechtigte Liquidatoren. 
	\item Bei Auflösung oder Aufhebung des Vereins oder bei Wegfall steuerbegünstigter Zwecke, fällt das Vermögen des Vereins einer im Rahmen des Auflösungsbeschlusses zu bestimmenden gemeinnützigen Organisation zu,
deren Zweck den Zwecken des Software Campus Alumni Vereins nahe steht.
	\item Die vorstehenden Bestimmungen gelten entsprechend, wenn dem Verein die Rechtsfähigkeit entzogen wurde.
	\end{enumerate}

\end{enumerate}

\end{document}